\documentclass{article}
\usepackage{amsmath}
\usepackage{amsfonts}
\usepackage{amssymb}

\title{Why Numerical Methods for a Linear Sea Level Equation?}
\author{Based on Kendall et al. (2005) and SLcode Implementation}
\date{}

\begin{document}
\maketitle

\section{The Fundamental Geophysical Problem}

Following Kendall et al. (2005), we start with the governing equations for gravitationally self-consistent sea level change:

\subsection{The Sea Level Definition}
Sea level is fundamentally controlled by gravity - water flows to form a surface where gravitational potential is constant (like water in a bathtub finding its level, but on a planetary scale with varying gravity).

\begin{equation}
S(\theta,\psi, t) = G(\theta,\psi, t) - N(\theta,\psi, t)
\end{equation}

where:
\begin{itemize}
\item $S(\theta,\psi, t)$ is the sea level (ocean depth when positive)
\item $G(\theta,\psi, t)$ is the height of the \textbf{gravitational equipotential surface} (the surface where ocean water naturally sits because gravity is constant)
\item $N(\theta,\psi, t)$ is the height of the \textbf{solid Earth surface} (bedrock/seafloor)
\item Both surfaces change when ice loads/unloads the Earth
\end{itemize}

\textbf{The Physical Picture:}
\begin{itemize}
\item Ice loads deform the solid Earth surface $N$ (makes it sink)
\item Ice loads also change Earth's gravity field, shifting the equipotential surface $G$
\item Water redistributes to stay on the new equipotential surface
\item Sea level $S$ is the gap between these two moving surfaces
\end{itemize}

The key insight is that both $G$ and $N$ depend on the total surface load, which includes unknown water redistribution.

\subsection{The Generalized Sea Level Equation}
The ocean height change is given by (Kendall et al., 2005, Eq. 17):
\begin{equation}
\Delta S(\theta,\psi, t_j) = \Delta SL(\theta,\psi, t_j) C(\theta,\psi, t_j) \beta(\theta,\psi, t_j) - T(\theta,\psi, t_0)[C(\theta,\psi, t_j)\beta(\theta,\psi, t_j) - C(\theta,\psi, t_0)\beta(\theta,\psi, t_0)]
\end{equation}

where:
\begin{itemize}
\item $C(\theta,\psi, t)$ is the ocean function (1 in ocean, 0 on land)
\item $\beta(\theta,\psi, t)$ accounts for grounded ice (1 where no grounded ice, 0 elsewhere)
\item $T(\theta,\psi, t_0)$ is the initial topography
\end{itemize}

\subsection{Love Number Theory}
For a spherically symmetric Earth, the sea level change is computed using Love numbers (Kendall et al., 2005, Eq. B18):
\begin{equation}
SL_{\ell m}(t_j) = T_\ell E_\ell \left[\rho_I I^*_{\ell m}(t_j) + \rho_w S_{\ell m}(t_j)\right] + \text{time history terms}
\end{equation}

where $E_\ell = 1 + k_\ell^{el} - h_\ell^{el}$ and $T_\ell = \frac{4\pi a^3}{M_e(2\ell + 1)}$.

\subsection{The Self-Consistency Constraint}
The water load depends on the sea level solution itself:
\begin{equation}
\sigma_{water}(\theta, \psi) = \rho_w [S(\theta, \psi) - \overline{S}] \Omega(\theta, \psi)
\end{equation}

This creates the fundamental coupling that requires iteration.

\section{The Complete Coupled System}

\subsection{Surface Load Components}
The total surface load consists of ice and water contributions:
\begin{equation}
\sigma(\theta,\psi) = \sigma_{ice}(\theta,\psi) + \sigma_{water}(\theta,\psi)
\end{equation}

where the water load is determined by kinematics:
\begin{equation}
\sigma_{water}(\theta,\psi) = \rho_w S(\theta,\psi)
\end{equation}

\subsection{Elastic Earth Response}
The solid Earth surface deforms according to Hooke's law. For surface loads, this gives the radial displacement:
\begin{equation}
N(\theta,\psi) = \int\int G_{elastic}(\theta,\psi; \theta',\psi') \sigma(\theta',\psi') dA'
\end{equation}

where $G_{elastic}$ is the elastic Green's function (solution to the continuum elasticity PDE with delta function forcing).

\subsection{Gravitational Equipotential Condition}
The ocean surface sits on a gravitational equipotential surface. The gravitational potential is:
\begin{equation}
U(r) = G \int\int \frac{\sigma(\theta',\psi')}{|r-r'|} dA'
\end{equation}

The equipotential condition requires:
\begin{equation}
U(G(\theta,\psi), \theta, \psi) = \text{constant}
\end{equation}

This implicitly defines $G(\theta,\psi)$ as a functional of the surface load $\sigma$.

\subsection{The Master Equation for Sea Level}
Combining all components, we obtain the complete nonlinear integro-differential equation for sea level:

\begin{equation}
\boxed{
S(\theta,\psi) = \mathcal{G}_{grav}[\sigma_{ice} + \rho_w S] - \int\int G_{elastic}(\theta,\psi; \theta',\psi') [\sigma_{ice}(\theta',\psi') + \rho_w S(\theta',\psi')] dA'
}
\end{equation}

where $\mathcal{G}_{grav}[\cdot]$ is the operator that solves the equipotential condition (5) for the gravitational surface height.

\section{Why This System Requires Iteration}

Combining equations (2), (3), and (4):
\begin{align}
S(\theta, \phi) &= \mathcal{G}[\sigma_{total}] - \frac{1}{g}\mathcal{P}[\sigma_{total}] \\
&= \mathcal{L}[\sigma_{total}]
\end{align}

where $\mathcal{G}$ and $\mathcal{P}$ are Green's function operators for elastic deformation and gravitational potential.

But from equation (6):
\begin{equation}
\sigma_{total} = \sigma_{ice} + \rho_w [S - \overline{S}] \Omega
\end{equation}

Substituting:
\begin{equation}
S = \mathcal{L}[\sigma_{ice}] + \mathcal{L}[\rho_w (S - \overline{S}) \Omega]
\end{equation}

This is the self-consistency equation showing why $S$ appears on both sides!

\section{Spherical Harmonic Treatment}

The PDEs are solved using spherical harmonic expansion. For any field $f(\theta, \phi)$:
\begin{equation}
f(\theta, \phi) = \sum_{l=0}^{\infty} \sum_{m=-l}^{l} f_{lm} Y_l^m(\theta, \phi)
\end{equation}

\subsection{Green's Functions in Spectral Domain}
The elastic deformation operator becomes:
\begin{equation}
\mathcal{G}[\sigma] \rightarrow h_l^{el} \sigma_{lm}
\end{equation}

The gravitational potential operator becomes:
\begin{equation}
\mathcal{P}[\sigma] \rightarrow \frac{4\pi G R^2}{2l+1} k_l^{el} \sigma_{lm}
\end{equation}

where $h_l^{el}$ and $k_l^{el}$ are elastic Love numbers.

\subsection{The Spectral Sea Level Equation}
In spectral space, equation (10) becomes:
\begin{equation}
S_{lm} = \left(h_l^{el} - \frac{4\pi G R^2 k_l^{el}}{g(2l+1)}\right) \sigma_{lm}^{total}
\end{equation}

Let $\beta_l = h_l^{el} - \frac{4\pi G R^2 k_l^{el}}{g(2l+1)}$, so:
\begin{equation}
S_{lm} = \beta_l \sigma_{lm}^{total}
\end{equation}

\section{Matrix Formulation of the Fixed-Point Problem}

The water load constraint in spectral space requires a transform back to spatial domain:
\begin{equation}
\sigma_{water}(\theta, \phi) = \rho_w [S(\theta, \phi) - \overline{S}] \Omega(\theta, \phi)
\end{equation}

This creates coupling between all spherical harmonic modes due to the product $S \cdot \Omega$.

\subsection{The Discrete System}
Discretizing on a global grid with $N$ points, we can write:
\begin{equation}
\mathbf{S} = \mathbf{B} \boldsymbol{\sigma}_{ice} + \mathbf{A} \mathbf{S}
\end{equation}

where:
\begin{itemize}
\item $\mathbf{S}$ is the $N \times 1$ sea level vector
\item $\boldsymbol{\sigma}_{ice}$ is the prescribed ice load
\item $\mathbf{B}$ represents the direct ice-to-sea-level response
\item $\mathbf{A}$ represents the water redistribution feedback
\end{itemize}

\subsection{Rearranging to Standard Form}
\begin{equation}
(\mathbf{I} - \mathbf{A}) \mathbf{S} = \mathbf{B} \boldsymbol{\sigma}_{ice}
\end{equation}

This is the linear system that could theoretically be solved directly.

The complexity arises from the \textbf{self-consistency requirement}. The total surface load consists of:
\begin{equation}
L_{total} = L_{ice} + L_{water}
\end{equation}

where:
\begin{itemize}
\item $L_{ice}$ is the prescribed ice load change (known)
\item $L_{water}$ depends on the sea level solution $S$ (unknown!)
\end{itemize}

\section{The Coupled System}

Both $N$ and $U$ depend on the total load:
\begin{align}
N(\theta, \phi) &= \sum_{l=0}^{\infty} \sum_{m=-l}^{l} h_l^{el} L_{lm} Y_l^m(\theta, \phi) \\
U(\theta, \phi) &= \frac{4\pi G \rho_{avg} R^2}{2l+1} \sum_{l=0}^{\infty} \sum_{m=-l}^{l} k_l^{el} L_{lm} Y_l^m(\theta, \phi)
\end{align}

where $L_{lm}$ are the spherical harmonic coefficients of the total load, including the unknown water redistribution.

\section{Water Load Redistribution}

The water load change is determined by:
\begin{equation}
L_{water}(\theta, \phi) = \rho_w [S(\theta, \phi) - \overline{S}] \cdot \Omega(\theta, \phi)
\end{equation}

where:
\begin{itemize}
\item $\rho_w$ is water density
\item $\overline{S}$ is the global mean sea level change
\item $\Omega(\theta, \phi)$ is the ocean function (1 in ocean, 0 on land)
\end{itemize}

\section{The Fixed-Point Problem}

Substituting everything together, we get:
\begin{equation}
S = \mathcal{F}[S]
\end{equation}

where $\mathcal{F}$ is a linear operator that depends on the solution $S$ itself through the water redistribution.

\section{Why Iteration is Required}

This creates a fixed-point equation that requires iterative solution:
\begin{align}
S^{(0)} &= \text{initial guess} \\
S^{(k+1)} &= \mathcal{F}[S^{(k)}] \quad \text{for } k = 0, 1, 2, \ldots
\end{align}

Convergence occurs when:
\begin{equation}
||S^{(k+1)} - S^{(k)}|| < \epsilon
\end{equation}

\section{Matrix Formulation (Why Direct Solve is Hard)}

In principle, we could write this as a linear system:
\begin{equation}
[I - A] \mathbf{S} = \mathbf{b}
\end{equation}

where:
\begin{itemize}
\item $\mathbf{S}$ is the vectorized sea level field
\item $\mathbf{b}$ represents the ice-only contribution
\item $A$ captures the water redistribution feedback
\end{itemize}

But this requires:
\begin{itemize}
\item Inverting a dense $N \times N$ matrix where $N \sim 10^4$ (global grid points)
\item The matrix $A$ involves spherical harmonic transforms and convolutions
\item Memory requirements: $O(N^2) \sim 10^8$ elements
\item Computational cost: $O(N^3) \sim 10^{12}$ operations
\end{itemize}

\section{Iterative Advantages}

The iterative approach:
\begin{itemize}
\item Uses fast spherical harmonic transforms: $O(N \log N)$ per iteration
\item Requires only $O(N)$ memory storage
\item Typically converges in 5-10 iterations
\item Total cost: $O(k \cdot N \log N)$ where $k \ll N$
\end{itemize}

\section{Spherical Harmonic Complexity}

Even if we could solve analytically, the expressions involve:
\begin{equation}
S(\theta, \phi) = \sum_{l=0}^{L_{max}} \sum_{m=-l}^{l} S_{lm} Y_l^m(\theta, \phi)
\end{equation}

For $L_{max} = 64$, this means $\sim 4225$ coefficients, each requiring:
\begin{itemize}
\item Love number applications: $h_l^{el}, k_l^{el}$
\item Convolution with load coefficients
\item Associated Legendre polynomial evaluations
\item Complex arithmetic for phase relationships
\end{itemize}

\section{Conclusion}

While the physics is linear, the self-consistency constraint creates a fixed-point problem that requires numerical iteration. The global scale and spherical harmonic complexity make analytical solutions impractical, even though the underlying mathematics is linear.

The apparent contradiction resolves as: \textbf{linear physics + self-consistent constraint + global scale = numerical necessity}.

\end{document}