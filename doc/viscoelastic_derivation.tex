\documentclass{article}
\usepackage{amsmath}
\usepackage{amsfonts}
\usepackage{amssymb}
\usepackage{physics}

\title{Viscoelastic Earth Response Theory: From Tromp \& Mitrovica (1999) to Sea Level Applications}
\author{Following the Normal-Mode Formalism for SLcode Implementation}
\date{}

\begin{document}
\maketitle

\section{Introduction: The Theoretical Foundation}

This document presents the complete theoretical framework underlying viscoelastic Earth response calculations in SLcode, following the rigorous normal-mode formalism developed by Tromp \& Mitrovica (1999) and applied to sea level problems by Kendall et al. (2005).

\textbf{Key Insight:} The exponential time dependence in viscoelastic Love numbers arises naturally from the normal-mode expansion of the Earth's response to surface loading, where each mode has its own characteristic decay time determined by the Earth's rheological structure.

\section{Physical Picture: Why Normal Modes?}

When a surface load is applied to a viscoelastic Earth:
\begin{enumerate}
\item The Earth responds with both \textbf{instantaneous elastic deformation} and \textbf{time-dependent viscous flow}
\item The viscous response can be decomposed into \textbf{normal modes} - fundamental oscillatory solutions of the viscoelastic Earth
\item Each normal mode has a characteristic \textbf{decay rate} $s_k$ that depends on the Earth's rheological structure
\item The total response is a \textbf{superposition} of all these modes, each decaying exponentially as $e^{-s_k t}$
\end{enumerate}

This is analogous to how a damped mechanical system responds to forcing - the response is a sum of damped oscillatory modes.

\section{The Viscoelastic Normal-Mode Problem (Tromp \& Mitrovica 1999)}

\subsection{Governing Equations}

For a self-gravitating, viscoelastic Earth, the fundamental equations are:

\textbf{1. Momentum balance (quasi-static):}
\begin{equation}
\nabla \cdot \mathbf{T} - \nabla(\rho \mathbf{u} \cdot \nabla \Phi) + \rho \nabla \phi + \rho_1 \nabla \Phi = 0
\end{equation}
where $\mathbf{T}$ is the incremental Cauchy stress, $\mathbf{u}$ is displacement, $\phi$ is perturbed gravitational potential, and $\Phi$ is the background potential.

\textbf{2. Maxwell rheology (in Laplace domain):}
\begin{equation}
\mathbf{T} = \kappa(\nabla \cdot \mathbf{u})\mathbf{I} + 2\mu(s)\mathbf{D}
\end{equation}
where $\mathbf{D}$ is the strain deviator and
\begin{equation}
\mu(s) = \frac{\mu s}{s + \mu/\eta}
\end{equation}
is the Laplace-transformed shear modulus for Maxwell rheology.

\textbf{3. Gravitational field equation:}
\begin{equation}
\nabla^2 \phi = 4\pi G \rho_1
\end{equation}
where $\rho_1$ is the perturbed density.

\subsection{Normal-Mode Eigenvalue Problem}

The viscoelastic normal modes $\{\mathbf{u}_k, \phi_k\}$ with decay rates $s_k$ satisfy:
\begin{equation}
\mathcal{L}(s_k) \mathbf{u}_k = 0, \quad \mathcal{L}_\phi \phi_k = 0
\end{equation}
where $\mathcal{L}(s)$ is the viscoelastic linear operator that depends on the Laplace variable $s$.

\textbf{Key Point:} The eigenvalues $s_k$ are the \textbf{characteristic decay rates} of the Earth's viscoelastic response. The corresponding relaxation times are:
\begin{equation}
\tau_k = \frac{1}{s_k}
\end{equation}

\subsection{Biorthogonality Relations}

A key feature of viscoelastic normal modes is that they satisfy \textbf{biorthogonality relations} (Tromp \& Mitrovica 1999, Section 5):

For distinct modes $\{\mathbf{u}_k, \phi_k\}$ and $\{\mathbf{u}_{k'}, \phi_{k'}\}$:
\begin{equation}
[\mathbf{u}_k, \phi_k; \{\mathcal{L}(s_k) - \mathcal{L}(s_{k'})\}\mathbf{u}_{k'}, 0] = 0
\end{equation}

where $[\cdot; \cdot]$ denotes the duality product. These relations are essential for constructing the Green's tensor.

\section{Surface-Load Response via Green's Tensor (Tromp \& Mitrovica 1999, Section 7)}

\subsection{Green's Tensor Construction}

The displacement response to a point force is expressed as a normal-mode expansion:
\begin{equation}
\mathbf{G}(\mathbf{r}, \mathbf{r}'; t) = -\sum_k \frac{1}{2s_k} \mathbf{u}_k(\mathbf{r}) \mathbf{u}_k(\mathbf{r}') e^{s_k t} H(t)
\end{equation}

where $H(t)$ is the Heaviside function.

\textbf{Physical Interpretation:} Each normal mode $k$ contributes to the Green's tensor with:
\begin{itemize}
\item \textbf{Spatial pattern}: $\mathbf{u}_k(\mathbf{r}) \mathbf{u}_k(\mathbf{r}')$ (eigenfunction product)
\item \textbf{Temporal evolution}: $e^{s_k t}$ (exponential decay)
\item \textbf{Amplitude}: $1/2s_k$ (inversely proportional to decay rate)
\end{itemize}

\subsection{Surface-Load Green's Vector}

For surface loading, the relevant response function is:
\begin{equation}
\boldsymbol{\Gamma}(\mathbf{r}, \mathbf{r}'; t) = \text{Re} \sum_k \frac{1}{2s_k} \mathbf{u}_k(\mathbf{r})[\mathbf{u}_k(\mathbf{r}') \cdot \nabla' \Phi(\mathbf{r}') + \phi_k(\mathbf{r}')] e^{s_k t} H(t)
\end{equation}

The displacement field due to an arbitrary surface load $\sigma(\mathbf{r}', t')$ is:
\begin{equation}
\mathbf{u}(\mathbf{r}, t) = \int_{-\infty}^t \int_{\partial V} \sigma(\mathbf{r}', t') \boldsymbol{\Gamma}(\mathbf{r}, \mathbf{r}'; t-t') d^2\mathbf{r}' dt'
\end{equation}

\section{Spherically Symmetric Earth: Love Numbers from Normal Modes}

\subsection{Spherical Harmonic Decomposition}

For a spherically symmetric Earth, the normal modes separate by spherical harmonic degree $\ell$:
\begin{align}
\mathbf{u}_k(\mathbf{r}) &= \mathbf{u}_{\ell,n}(r) Y_\ell^m(\theta, \phi) \\
\phi_k(\mathbf{r}) &= \phi_{\ell,n}(r) Y_\ell^m(\theta, \phi)
\end{align}

where $n$ indexes the radial modes for each degree $\ell$.

The decay rates become $s_{\ell,n}$, giving degree-dependent relaxation times:
\begin{equation}
\tau_{\ell,n} = \frac{1}{s_{\ell,n}}
\end{equation}

\subsection{Love Number Construction}

The \textbf{viscoelastic Love numbers} are constructed from the surface values of the radial eigenfunctions:

\begin{align}
h_\ell(t) &= h_\ell^E \delta(t) + \sum_{n=1}^{\infty} r_{\ell,n}^h e^{-t/\tau_{\ell,n}} \\
k_\ell(t) &= k_\ell^E \delta(t) + \sum_{n=1}^{\infty} r_{\ell,n}^k e^{-t/\tau_{\ell,n}}
\end{align}

where:
\begin{itemize}
\item $h_\ell^E, k_\ell^E$ are \textbf{elastic Love numbers} (instantaneous response)
\item $r_{\ell,n}^{h,k}$ are \textbf{residues} (amplitudes) determined from the normal-mode eigenfunctions
\item $\tau_{\ell,n} = 1/s_{\ell,n}$ are \textbf{relaxation times} from the normal-mode eigenvalues
\end{itemize}

\textbf{Key Result:} The exponential terms $e^{-t/\tau_{\ell,n}}$ arise directly from the eigenvalue spectrum $s_{\ell,n}$ of the viscoelastic normal-mode problem.

\section{Physical Interpretation of Normal Modes}

\subsection{Spherical Harmonic Degrees}
\begin{itemize}
\item \textbf{$\ell = 0$}: Radial expansion/contraction (breathing mode)
\item \textbf{$\ell = 1$}: Translational motion (center of mass)
\item \textbf{$\ell = 2$}: Flattening/elongation (tidal deformation)
\item \textbf{Higher $\ell$}: Shorter wavelength surface deformation
\end{itemize}

\subsection{Radial Mode Structure}

Each degree $\ell$ has multiple relaxation times $\tau_{\ell,n}$ corresponding to different \textbf{radial modes}:
\begin{itemize}
\item \textbf{Fast modes ($n=1,2,...$)}: Surface/shallow deformation (years to decades)
\item \textbf{Slow modes ($n$ large)}: Deep mantle flow (thousands of years)
\end{itemize}

The radial mode index $n$ reflects the number of nodes in the radial eigenfunction - higher $n$ corresponds to more complex radial structure and typically longer relaxation times.

\section{Connection to Kendall et al. (2005): Time-Dependent Sea Level Equations}

\subsection{From Normal Modes to Sea Level Response}

Kendall et al. (2005) applied the Tromp \& Mitrovica (1999) framework to sea level problems. The time-dependent sea level response in spherical harmonic space is:

\begin{equation}
SL_{\ell m}(t_j) = T_\ell E_\ell[\rho_I I^*_{\ell m}(t_j) + \rho_w S_{\ell m}(t_j)] + T_\ell \sum_{i=1}^{j-1} \Delta t \sum_k A_\ell^k[\rho_I \Delta I^*_{\ell m}(t_i) + \rho_w \Delta S_{\ell m}(t_i)] e^{-(t_j-t_i)/\tau_k}
\end{equation}

where:
\begin{itemize}
\item $E_\ell = 1 + k_\ell^{el} - h_\ell^{el}$ (elastic response coefficient)
\item $T_\ell = \frac{4\pi a^3}{M_e(2\ell + 1)}$ (normalization factor)
\item $A_\ell^k$ are the normal-mode amplitudes from Tromp (1999)
\item $\tau_k = 1/s_k$ are the normal-mode relaxation times
\item The exponential terms $e^{-(t_j-t_i)/\tau_k}$ encode the **viscoelastic memory**
\end{itemize}

\subsection{Implementation in SLcode: MATLAB vs Python}

\textbf{Key Implementation Note:} The complete Tromp (1999) → Kendall (2005) framework is implemented differently across SLcode versions:

\subsubsection{MATLAB Implementation (Complete Viscoelastic Theory)}

The MATLAB files ({\tt SL\_equation\_viscoelastic\_*.m}) implement the \textbf{full Tromp (1999) normal-mode theory}:

\begin{equation}
\boxed{
\beta_\ell(t) = \sum_{k=1}^{N_{modes}} \frac{(k_{amp,\ell,k} - h_{amp,\ell,k})}{s_{\ell,k}} \left(1 - e^{-s_{\ell,k} \cdot t}\right)
}
\end{equation}

where:
\begin{itemize}
\item {\tt spoles($\ell$,k)} $= s_{\ell,k}$ (normal-mode decay rates)
\item {\tt k\_amp($\ell$,k)}, {\tt h\_amp($\ell$,k)} = amplitude coefficients from normal-mode eigenfunctions
\item {\tt mode\_found($\ell$)} = number of computed modes for degree $\ell$
\end{itemize}

The MATLAB code loads these parameters from precomputed normal-mode solutions:
\begin{verbatim}
load SavedLN/prem.l90C.umVM2.lmVM2.mat
% Contains: spoles, k_amp, h_amp, k_amp_tide, h_amp_tide, mode_found
\end{verbatim}

\subsubsection{Python Implementation (Elastic Only)}

The Python files ({\tt SL\_equation\_elastic.py}) implement \textbf{only the elastic limit}:

\begin{equation}
E_{ml} = 1 + k_{lm} - h_{lm}
\end{equation}

using static Love numbers without time dependence. The Python version:
\begin{itemize}
\item  Loads elastic Love numbers: {\tt love['h\_el']}, {\tt love['k\_el']}
\item  Implements Kendall (2005) elastic sea level equation
\item  \textbf{Does not include} {\tt spoles}, {\tt k\_amp}, {\tt h\_amp} parameters
\item  \textbf{No viscoelastic time dependence} or normal-mode theory
\end{itemize}

\textbf{Status:} According to the repository documentation, implementing the full viscoelastic Python version is a "Long Term" development goal.

\section{Mathematical Derivation: From Continuum Mechanics to Implementation}

\subsection{The Complete Theoretical Chain}

The complete progression from fundamental physics to numerical implementation follows:

\begin{center}
\textbf{Maxwell Rheology} \\
$\downarrow$ \\
\textbf{Viscoelastic Normal-Mode Problem} \\
$\downarrow$ \\
\textbf{Eigenvalue Spectrum} $\{s_{\ell,k}\}$ \\
$\downarrow$ \\
\textbf{Green's Tensor Construction} \\
$\downarrow$ \\
\textbf{Love Number Time Dependence} \\
$\downarrow$ \\
\textbf{Kendall (2005) Sea Level Equations} \\
$\downarrow$ \\
\textbf{SLcode MATLAB Implementation}
\end{center}

\subsection{Physical Significance of Normal-Mode Parameters}

The parameters in SLcode MATLAB files have direct physical meaning:

\begin{itemize}
\item \textbf{{\tt spoles($\ell$,k)} = $s_{\ell,k}$}: Characteristic decay rates of Earth's viscoelastic response
  \begin{itemize}
  \item Determined by mantle viscosity structure and elastic moduli
  \item Fast modes: $s_{\ell,k} \sim 10^{-1}$ yr$^{-1}$ (surface/lithosphere)
  \item Slow modes: $s_{\ell,k} \sim 10^{-4}$ yr$^{-1}$ (deep mantle flow)
  \end{itemize}

\item \textbf{{\tt k\_amp($\ell$,k)}, {\tt h\_amp($\ell$,k)}}: Amplitude coefficients from surface values of normal-mode eigenfunctions
  \begin{itemize}
  \item Related to gravitational potential and radial displacement eigenfunctions
  \item Encode the coupling between loading and deformation for each mode
  \end{itemize}

\item \textbf{{\tt mode\_found($\ell$)}}: Number of computed normal modes for each spherical harmonic degree
  \begin{itemize}
  \item Higher $\ell$ typically requires more modes for convergence
  \item Reflects the complexity of radial eigenfunction structure
  \end{itemize}
\end{itemize}

\subsection{Displacement Field Expansion}

The displacement field separates as:
\begin{align}
u_r(r,\theta,\phi,t) &= \sum_{\ell=0}^{\infty} \sum_{m=-\ell}^{\ell} u_r^{\ell m}(r,t) Y_\ell^m(\theta,\phi) \\
u_\theta(r,\theta,\phi,t) &= \sum_{\ell=0}^{\infty} \sum_{m=-\ell}^{\ell} u_\theta^{\ell m}(r,t) \frac{\partial Y_\ell^m}{\partial \theta} \\
u_\phi(r,\theta,\phi,t) &= \sum_{\ell=0}^{\infty} \sum_{m=-\ell}^{\ell} u_\phi^{\ell m}(r,t) \frac{1}{\sin\theta}\frac{\partial Y_\ell^m}{\partial \phi}
\end{align}

\subsection{Spherical Laplacian Eigenvalue Property}

The key insight is that spherical harmonics are eigenfunctions of the angular part of $\nabla^2$:
\begin{equation}
\nabla^2 Y_\ell^m = \left[\frac{1}{r^2}\frac{\partial}{\partial r}\left(r^2\frac{\partial}{\partial r}\right) - \frac{\ell(\ell+1)}{r^2}\right] Y_\ell^m
\end{equation}

\subsection{Separation by Degree}

When we substitute the expansion (18)-(20) into the linear operator $L_{ij}$, **the problem separates by spherical harmonic degree $\ell$**.

For each $\ell$, we get a **radial ODE system**:
\begin{equation}
\mathcal{L}_\ell^{(r)} \begin{pmatrix} u_r^{\ell m}(r) \\ u_\theta^{\ell m}(r) \end{pmatrix} = \lambda_{\ell,k} \begin{pmatrix} u_r^{\ell m}(r) \\ u_\theta^{\ell m}(r) \end{pmatrix}
\end{equation}

where $\mathcal{L}_\ell^{(r)}$ is the **radial differential operator** that depends on degree $\ell$.

\subsection{Radial Operator Structure}

The radial operator has the form:
\begin{equation}
\mathcal{L}_\ell^{(r)} = \begin{pmatrix}
\frac{d^2}{dr^2} + \frac{2}{r}\frac{d}{dr} - \frac{\ell(\ell+1)}{r^2} + \frac{\mu}{\eta} & \text{coupling terms} \\
\text{coupling terms} & \frac{d^2}{dr^2} + \frac{2}{r}\frac{d}{dr} - \frac{\ell(\ell+1)}{r^2} + \frac{\mu}{\eta}
\end{pmatrix}
\end{equation}

\section{Connection to Love Numbers}

\subsection{Eigenvalue Spectrum}

For each degree $\ell$, the radial operator $\mathcal{L}_\ell^{(r)}$ has eigenvalues:
\begin{equation}
\lambda_{\ell,1}, \lambda_{\ell,2}, \lambda_{\ell,3}, \ldots
\end{equation}

These give the **relaxation times** for degree $\ell$:
\begin{equation}
\tau_{\ell,k} = \frac{1}{\lambda_{\ell,k}}
\end{equation}

\subsection{Love Number Time Dependence}

The **viscoelastic Love numbers** are constructed from these eigenvalues:
\begin{align}
h_\ell(t) &= h_\ell^E \delta(t) + \sum_{k=1}^{\infty} r_{\ell,k}^h e^{-t/\tau_{\ell,k}} \\
k_\ell(t) &= k_\ell^E \delta(t) + \sum_{k=1}^{\infty} r_{\ell,k}^k e^{-t/\tau_{\ell,k}}
\end{align}

where:
\begin{itemize}
\item $h_\ell^E, k_\ell^E$ are **elastic Love numbers** (instantaneous response)
\item $r_{\ell,k}^{h,k}$ are **residues** (amplitudes of each mode)
\item $\tau_{\ell,k} = 1/\lambda_{\ell,k}$ are **relaxation times** from eigenvalues
\end{itemize}

\subsection{Physical Interpretation}

\begin{itemize}
\item **$\ell = 0$**: Radial expansion/contraction (breathing mode)
\item **$\ell = 1$**: Translational motion (center of mass)
\item **$\ell = 2$**: Flattening/elongation (tidal deformation)
\item **Higher $\ell$**: Shorter wavelength surface deformation
\end{itemize}

Each degree $\ell$ has multiple relaxation times $\tau_{\ell,k}$ corresponding to different **radial modes**:
\begin{itemize}
\item Fast modes: Surface/shallow deformation (years to decades)
\item Slow modes: Deep mantle flow (thousands of years)
\end{itemize}

\section{Summary}

The complete chain is:
\begin{center}
\textbf{Maxwell rheology} $\rightarrow$ \textbf{Linear operator $L$} $\rightarrow$ \textbf{Spherical harmonic separation} $\rightarrow$ \textbf{Radial eigenvalue problems} $\rightarrow$ \textbf{Relaxation times $\tau_{\ell,k}$} $\rightarrow$ \textbf{Love number exponentials}
\end{center}

The exponential terms $e^{-t/\tau_{\ell,k}}$ in viscoelastic Love numbers arise directly from the **eigenvalue spectrum** of the radial viscoelastic operator for each spherical harmonic degree $\ell$.

\end{document}
