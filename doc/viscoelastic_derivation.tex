\documentclass{article}
\usepackage{amsmath}
\usepackage{amsfonts}
\usepackage{amssymb}

\title{Derivation of the Viscoelastic Linear Operator}
\author{From Basic Equations to Eigenvalue Problem}
\date{}

\begin{document}
\maketitle

\section{Complete System of Equations}

We start with the four fundamental equations for viscoelastic deformation:

\textbf{1. Momentum balance (quasi-static):}
\begin{equation}
\nabla \cdot \sigma = -f
\end{equation}
where $f$ is body force (gravitational loading).

\textbf{2. Strain-displacement relation:}
\begin{equation}
\epsilon_{ij} = \frac{1}{2}\left(\frac{\partial u_i}{\partial x_j} + \frac{\partial u_j}{\partial x_i}\right)
\end{equation}

\textbf{3. Maxwell rheology:}
\begin{equation}
\dot{\epsilon}_{ij} = \frac{1}{2\eta}\sigma_{ij}' + \frac{1}{2\mu}\dot{\sigma}_{ij}'
\end{equation}
where $\sigma_{ij}' = \sigma_{ij} - \frac{1}{3}\sigma_{kk}\delta_{ij}$ is deviatoric stress.

\textbf{4. Elastic stress-strain relation:}
\begin{equation}
\sigma_{ij} = 2\mu\epsilon_{ij} + \lambda\epsilon_{kk}\delta_{ij}
\end{equation}
where $\lambda, \mu$ are Lamé parameters.

\section{Elimination Process}

\subsection{Step 1: Express stress in terms of displacement}

From equations (2) and (4):
\begin{equation}
\sigma_{ij} = \mu\left(\frac{\partial u_i}{\partial x_j} + \frac{\partial u_j}{\partial x_i}\right) + \lambda\frac{\partial u_k}{\partial x_k}\delta_{ij}
\end{equation}

\subsection{Step 2: Deviatoric stress}

The deviatoric part of (5):
\begin{equation}
\sigma_{ij}' = \mu\left(\frac{\partial u_i}{\partial x_j} + \frac{\partial u_j}{\partial x_i}\right) - \frac{2\mu}{3}\frac{\partial u_k}{\partial x_k}\delta_{ij}
\end{equation}

\subsection{Step 3: Time derivatives}

From (2):
\begin{equation}
\dot{\epsilon}_{ij} = \frac{1}{2}\left(\frac{\partial \dot{u}_i}{\partial x_j} + \frac{\partial \dot{u}_j}{\partial x_i}\right)
\end{equation}

From (6):
\begin{equation}
\dot{\sigma}_{ij}' = \mu\left(\frac{\partial \dot{u}_i}{\partial x_j} + \frac{\partial \dot{u}_j}{\partial x_i}\right) - \frac{2\mu}{3}\frac{\partial \dot{u}_k}{\partial x_k}\delta_{ij}
\end{equation}

\section{Substitution into Maxwell Equation}

Substituting (7) and (8) into the Maxwell rheology (3):

\begin{equation}
\frac{1}{2}\left(\frac{\partial \dot{u}_i}{\partial x_j} + \frac{\partial \dot{u}_j}{\partial x_i}\right) = \frac{1}{2\eta}\left[\mu\left(\frac{\partial u_i}{\partial x_j} + \frac{\partial u_j}{\partial x_i}\right) - \frac{2\mu}{3}\frac{\partial u_k}{\partial x_k}\delta_{ij}\right] + \frac{1}{2\mu}\left[\mu\left(\frac{\partial \dot{u}_i}{\partial x_j} + \frac{\partial \dot{u}_j}{\partial x_i}\right) - \frac{2\mu}{3}\frac{\partial \dot{u}_k}{\partial x_k}\delta_{ij}\right]
\end{equation}

Simplifying:
\begin{equation}
\frac{\partial \dot{u}_i}{\partial x_j} + \frac{\partial \dot{u}_j}{\partial x_i} = \frac{\mu}{\eta}\left(\frac{\partial u_i}{\partial x_j} + \frac{\partial u_j}{\partial x_i}\right) - \frac{2\mu}{3\eta}\frac{\partial u_k}{\partial x_k}\delta_{ij} + \frac{\partial \dot{u}_i}{\partial x_j} + \frac{\partial \dot{u}_j}{\partial x_i} - \frac{2}{3}\frac{\partial \dot{u}_k}{\partial x_k}\delta_{ij}
\end{equation}

This gives us:
\begin{equation}
0 = \frac{\mu}{\eta}\left(\frac{\partial u_i}{\partial x_j} + \frac{\partial u_j}{\partial x_i}\right) - \frac{2\mu}{3\eta}\frac{\partial u_k}{\partial x_k}\delta_{ij} - \frac{2}{3}\frac{\partial \dot{u}_k}{\partial x_k}\delta_{ij}
\end{equation}

\section{The Linear Operator}

From momentum balance (1) with stress (5):
\begin{equation}
\mu\nabla^2 u_i + (\lambda + \mu)\frac{\partial}{\partial x_i}\left(\frac{\partial u_k}{\partial x_k}\right) = -f_i
\end{equation}

Combined with the Maxwell constraint (11), we get the \textbf{viscoelastic wave equation}:

\begin{equation}
\boxed{
\mu\nabla^2 \dot{u}_i + (\lambda + \mu)\frac{\partial}{\partial x_i}\left(\frac{\partial \dot{u}_k}{\partial x_k}\right) + \frac{\mu^2}{\eta}\nabla^2 u_i + \frac{\mu(\lambda + \mu)}{\eta}\frac{\partial}{\partial x_i}\left(\frac{\partial u_k}{\partial x_k}\right) = -\dot{f}_i
}
\end{equation}

\section{Eigenvalue Problem}

For time-harmonic solutions $u_i = \hat{u}_i e^{-\lambda t}$, equation (13) becomes:

\begin{equation}
\left[-\lambda\mu\nabla^2 + \frac{\mu^2}{\eta}\nabla^2\right]\hat{u}_i + \left[-\lambda(\lambda + \mu)\frac{\partial}{\partial x_i}\frac{\partial}{\partial x_k} + \frac{\mu(\lambda + \mu)}{\eta}\frac{\partial}{\partial x_i}\frac{\partial}{\partial x_k}\right]\hat{u}_k = \lambda\hat{f}_i
\end{equation}

This gives us the \textbf{linear operator}:

\begin{equation}
\boxed{
L_{ij}[\hat{u}] = \left[\mu\left(\frac{\mu}{\eta} - \lambda\right)\nabla^2\delta_{ij} + (\lambda + \mu)\left(\frac{\mu}{\eta} - \lambda\right)\frac{\partial^2}{\partial x_i \partial x_j}\right]\hat{u}_j
}
\end{equation}

\section{Eigenvalue Interpretation}

The eigenvalue equation is:
\begin{equation}
L_{ij}[\hat{u}_j] = \lambda\hat{f}_i
\end{equation}

The eigenvalues $\lambda$ satisfy:
\begin{equation}
\lambda = \frac{\mu}{\eta} \pm \text{corrections from elastic terms}
\end{equation}

These eigenvalues determine the \textbf{relaxation times} $\tau = 1/\lambda$ that appear in the viscoelastic Love numbers:
\begin{equation}
h_\ell(t) = h_\ell^E\delta(t) + \sum_k r_\ell^k e^{-t/\tau_k}
\end{equation}

\section{Spherical Harmonic Decomposition}

For a **spherically symmetric Earth**, we transform to spherical coordinates $(r,\theta,\phi)$ and expand in spherical harmonics.

\subsection{Displacement Field Expansion}

The displacement field separates as:
\begin{align}
u_r(r,\theta,\phi,t) &= \sum_{\ell=0}^{\infty} \sum_{m=-\ell}^{\ell} u_r^{\ell m}(r,t) Y_\ell^m(\theta,\phi) \\
u_\theta(r,\theta,\phi,t) &= \sum_{\ell=0}^{\infty} \sum_{m=-\ell}^{\ell} u_\theta^{\ell m}(r,t) \frac{\partial Y_\ell^m}{\partial \theta} \\
u_\phi(r,\theta,\phi,t) &= \sum_{\ell=0}^{\infty} \sum_{m=-\ell}^{\ell} u_\phi^{\ell m}(r,t) \frac{1}{\sin\theta}\frac{\partial Y_\ell^m}{\partial \phi}
\end{align}

\subsection{Spherical Laplacian Eigenvalue Property}

The key insight is that spherical harmonics are eigenfunctions of the angular part of $\nabla^2$:
\begin{equation}
\nabla^2 Y_\ell^m = \left[\frac{1}{r^2}\frac{\partial}{\partial r}\left(r^2\frac{\partial}{\partial r}\right) - \frac{\ell(\ell+1)}{r^2}\right] Y_\ell^m
\end{equation}

\subsection{Separation by Degree}

When we substitute the expansion (18)-(20) into the linear operator $L_{ij}$, **the problem separates by spherical harmonic degree $\ell$**.

For each $\ell$, we get a **radial ODE system**:
\begin{equation}
\mathcal{L}_\ell^{(r)} \begin{pmatrix} u_r^{\ell m}(r) \\ u_\theta^{\ell m}(r) \end{pmatrix} = \lambda_{\ell,k} \begin{pmatrix} u_r^{\ell m}(r) \\ u_\theta^{\ell m}(r) \end{pmatrix}
\end{equation}

where $\mathcal{L}_\ell^{(r)}$ is the **radial differential operator** that depends on degree $\ell$.

\subsection{Radial Operator Structure}

The radial operator has the form:
\begin{equation}
\mathcal{L}_\ell^{(r)} = \begin{pmatrix}
\frac{d^2}{dr^2} + \frac{2}{r}\frac{d}{dr} - \frac{\ell(\ell+1)}{r^2} + \frac{\mu}{\eta} & \text{coupling terms} \\
\text{coupling terms} & \frac{d^2}{dr^2} + \frac{2}{r}\frac{d}{dr} - \frac{\ell(\ell+1)}{r^2} + \frac{\mu}{\eta}
\end{pmatrix}
\end{equation}

\section{Connection to Love Numbers}

\subsection{Eigenvalue Spectrum}

For each degree $\ell$, the radial operator $\mathcal{L}_\ell^{(r)}$ has eigenvalues:
\begin{equation}
\lambda_{\ell,1}, \lambda_{\ell,2}, \lambda_{\ell,3}, \ldots
\end{equation}

These give the **relaxation times** for degree $\ell$:
\begin{equation}
\tau_{\ell,k} = \frac{1}{\lambda_{\ell,k}}
\end{equation}

\subsection{Love Number Time Dependence}

The **viscoelastic Love numbers** are constructed from these eigenvalues:
\begin{align}
h_\ell(t) &= h_\ell^E \delta(t) + \sum_{k=1}^{\infty} r_{\ell,k}^h e^{-t/\tau_{\ell,k}} \\
k_\ell(t) &= k_\ell^E \delta(t) + \sum_{k=1}^{\infty} r_{\ell,k}^k e^{-t/\tau_{\ell,k}}
\end{align}

where:
\begin{itemize}
\item $h_\ell^E, k_\ell^E$ are **elastic Love numbers** (instantaneous response)
\item $r_{\ell,k}^{h,k}$ are **residues** (amplitudes of each mode)
\item $\tau_{\ell,k} = 1/\lambda_{\ell,k}$ are **relaxation times** from eigenvalues
\end{itemize}

\subsection{Physical Interpretation}

\begin{itemize}
\item **$\ell = 0$**: Radial expansion/contraction (breathing mode)
\item **$\ell = 1$**: Translational motion (center of mass)
\item **$\ell = 2$**: Flattening/elongation (tidal deformation)
\item **Higher $\ell$**: Shorter wavelength surface deformation
\end{itemize}

Each degree $\ell$ has multiple relaxation times $\tau_{\ell,k}$ corresponding to different **radial modes**:
\begin{itemize}
\item Fast modes: Surface/shallow deformation (years to decades)
\item Slow modes: Deep mantle flow (thousands of years)
\end{itemize}

\section{Summary}

The complete chain is:
\begin{center}
\textbf{Maxwell rheology} $\rightarrow$ \textbf{Linear operator $L$} $\rightarrow$ \textbf{Spherical harmonic separation} $\rightarrow$ \textbf{Radial eigenvalue problems} $\rightarrow$ \textbf{Relaxation times $\tau_{\ell,k}$} $\rightarrow$ \textbf{Love number exponentials}
\end{center}

The exponential terms $e^{-t/\tau_{\ell,k}}$ in viscoelastic Love numbers arise directly from the **eigenvalue spectrum** of the radial viscoelastic operator for each spherical harmonic degree $\ell$.

\end{document}