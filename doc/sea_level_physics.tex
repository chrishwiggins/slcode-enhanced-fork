\documentclass{article}
\usepackage{amsmath}
\usepackage{amsfonts}
\usepackage{amssymb}

\title{Why Numerical Methods for Linear Sea Level Physics?}
\author{Rigorous Derivation}
\date{}

\begin{document}
\maketitle

\section{What This Is About}

\textbf{The Big Picture:} When ice sheets melt (Greenland, Antarctica), where does sea level rise? Not uniformly! The physics of gravity + Earth deformation creates complex spatial patterns.

\section{Basic Setup}

\textbf{Single-letter variables only:}
\begin{itemize}
\item $s(\theta,\phi,t)$ = sea level height (water column thickness)
\item $g(\theta,\phi,t)$ = gravitational equipotential surface height
\item $n(\theta,\phi,t)$ = solid Earth surface height (seafloor)
\item $\sigma(\theta,\phi,t)$ = surface mass density [kg/m²]
\item $\rho$ = density [kg/m³]
\end{itemize}

\section{Fundamental Physics}

\textbf{Kinematic definition (positive terms only):}
\begin{equation}
g = s + n
\end{equation}

\textbf{Physical meaning:} Height of water surface = thickness of water + height of seafloor

\textbf{Gravitational potential:}
\begin{equation}
u(r) = G \int\int \frac{\sigma(\theta',\phi')}{|r-r'|} da'
\end{equation}

\textbf{Equipotential condition:}
\begin{equation}
u(g(\theta,\phi), \theta, \phi) = c
\end{equation}
where $c$ is constant.

\textbf{Elastic deformation:}
\begin{equation}
n(\theta,\phi) = \int\int k(\theta,\phi; \theta',\phi') \sigma(\theta',\phi') da'
\end{equation}
where $k$ is elastic Green's function.

\section{Surface Load Decomposition}

Total surface load:
\begin{equation}
\sigma = \sigma_i + \sigma_w
\end{equation}

Water load from kinematics:
\begin{equation}
\sigma_w = \rho_w s
\end{equation}

Substituting into (5):
\begin{equation}
\sigma = \sigma_i + \rho_w s
\end{equation}

\section{The Coupled System}

From equations (3) and (4):
\begin{align}
g &= \mathcal{G}[\sigma_i + \rho_w s] \\
n &= \int\int k(\theta,\phi; \theta',\phi') [\sigma_i(\theta',\phi') + \rho_w s(\theta',\phi')] da'
\end{align}

where $\mathcal{G}[\cdot]$ solves the equipotential condition (3).

\section{Master Equation}

Substituting (8) and (9) into (1), solving for $s$:
\begin{equation}
\boxed{s = \mathcal{G}[\sigma_i + \rho_w s] - \int\int k(\theta,\phi; \theta',\phi') [\sigma_i(\theta',\phi') + \rho_w s(\theta',\phi')] da'}
\end{equation}

This is a \textbf{nonlinear integral equation} for $s(\theta,\phi)$.

\section{Why Iteration is Required}

Equation (10) has the form:
\begin{equation}
s = \mathcal{F}[s]
\end{equation}

where $\mathcal{F}$ is a nonlinear operator due to the equipotential condition.

\textbf{The physics is linear} (both gravity and elasticity are linear), but the \textbf{self-consistency constraint} creates nonlinearity:
\begin{itemize}
\item Water redistributes based on $s$
\item But $s$ depends on water distribution
\item Fixed-point iteration required: $s^{(k+1)} = \mathcal{F}[s^{(k)}]$
\end{itemize}

\section{Spherical Harmonic Approach}

Expand in spherical harmonics:
\begin{equation}
s(\theta,\phi) = \sum_{\ell m} s_{\ell m} Y_\ell^m(\theta,\phi)
\end{equation}

The operators become:
\begin{align}
\mathcal{G}[\sigma] &\rightarrow \beta_\ell^g \sigma_{\ell m} \\
\int\int k \sigma da' &\rightarrow \beta_\ell^e \sigma_{\ell m}
\end{align}

where $\beta_\ell^g, \beta_\ell^e$ are Love numbers.

\section{Discrete System}

In spectral space, equation (10) becomes:
\begin{equation}
s_{\ell m} = (\beta_\ell^g - \beta_\ell^e)[\sigma_{i,\ell m} + \rho_w s_{\ell m}]
\end{equation}

Rearranging:
\begin{equation}
s_{\ell m}[1 - \rho_w(\beta_\ell^g - \beta_\ell^e)] = (\beta_\ell^g - \beta_\ell^e)\sigma_{i,\ell m}
\end{equation}

\textbf{This looks solvable!} But the ocean function $\Omega(\theta,\phi)$ (where water can exist) couples all modes, requiring iteration.

\section{Conclusion}

The sea level equation is \textbf{linear in physics} but \textbf{nonlinear in practice} due to:
\begin{enumerate}
\item Self-consistent water redistribution
\item Ocean geometry constraints
\item Equipotential surface computation
\end{enumerate}

Numerical iteration is unavoidable despite linear underlying physics.

\end{document}